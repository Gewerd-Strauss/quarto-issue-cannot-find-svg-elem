% Options for packages loaded elsewhere
% Options for packages loaded elsewhere
\PassOptionsToPackage{unicode}{hyperref}
\PassOptionsToPackage{hyphens}{url}
\PassOptionsToPackage{dvipsnames,svgnames,x11names}{xcolor}
%
\documentclass[
]{article}
\usepackage{xcolor}
\usepackage[top=1.5cm,bottom=1.5cm,left=0.2cm,right=0.2cm,landscape]{geometry}
\usepackage{amsmath,amssymb}
\setcounter{secnumdepth}{5}
\usepackage{iftex}
\ifPDFTeX
  \usepackage[T1]{fontenc}
  \usepackage[utf8]{inputenc}
  \usepackage{textcomp} % provide euro and other symbols
\else % if luatex or xetex
  \usepackage{unicode-math} % this also loads fontspec
  \defaultfontfeatures{Scale=MatchLowercase}
  \defaultfontfeatures[\rmfamily]{Ligatures=TeX,Scale=1}
\fi
\usepackage{lmodern}
\ifPDFTeX\else
  % xetex/luatex font selection
\fi
% Use upquote if available, for straight quotes in verbatim environments
\IfFileExists{upquote.sty}{\usepackage{upquote}}{}
\IfFileExists{microtype.sty}{% use microtype if available
  \usepackage[]{microtype}
  \UseMicrotypeSet[protrusion]{basicmath} % disable protrusion for tt fonts
}{}
\makeatletter
\@ifundefined{KOMAClassName}{% if non-KOMA class
  \IfFileExists{parskip.sty}{%
    \usepackage{parskip}
  }{% else
    \setlength{\parindent}{0pt}
    \setlength{\parskip}{6pt plus 2pt minus 1pt}}
}{% if KOMA class
  \KOMAoptions{parskip=half}}
\makeatother
% Make \paragraph and \subparagraph free-standing
\makeatletter
\ifx\paragraph\undefined\else
  \let\oldparagraph\paragraph
  \renewcommand{\paragraph}{
    \@ifstar
      \xxxParagraphStar
      \xxxParagraphNoStar
  }
  \newcommand{\xxxParagraphStar}[1]{\oldparagraph*{#1}\mbox{}}
  \newcommand{\xxxParagraphNoStar}[1]{\oldparagraph{#1}\mbox{}}
\fi
\ifx\subparagraph\undefined\else
  \let\oldsubparagraph\subparagraph
  \renewcommand{\subparagraph}{
    \@ifstar
      \xxxSubParagraphStar
      \xxxSubParagraphNoStar
  }
  \newcommand{\xxxSubParagraphStar}[1]{\oldsubparagraph*{#1}\mbox{}}
  \newcommand{\xxxSubParagraphNoStar}[1]{\oldsubparagraph{#1}\mbox{}}
\fi
\makeatother


\usepackage{longtable,booktabs,array}
\usepackage{calc} % for calculating minipage widths
% Correct order of tables after \paragraph or \subparagraph
\usepackage{etoolbox}
\makeatletter
\patchcmd\longtable{\par}{\if@noskipsec\mbox{}\fi\par}{}{}
\makeatother
% Allow footnotes in longtable head/foot
\IfFileExists{footnotehyper.sty}{\usepackage{footnotehyper}}{\usepackage{footnote}}
\makesavenoteenv{longtable}
\usepackage{graphicx}
\makeatletter
\newsavebox\pandoc@box
\newcommand*\pandocbounded[1]{% scales image to fit in text height/width
  \sbox\pandoc@box{#1}%
  \Gscale@div\@tempa{\textheight}{\dimexpr\ht\pandoc@box+\dp\pandoc@box\relax}%
  \Gscale@div\@tempb{\linewidth}{\wd\pandoc@box}%
  \ifdim\@tempb\p@<\@tempa\p@\let\@tempa\@tempb\fi% select the smaller of both
  \ifdim\@tempa\p@<\p@\scalebox{\@tempa}{\usebox\pandoc@box}%
  \else\usebox{\pandoc@box}%
  \fi%
}
% Set default figure placement to htbp
\def\fps@figure{htbp}
\makeatother





\setlength{\emergencystretch}{3em} % prevent overfull lines

\providecommand{\tightlist}{%
  \setlength{\itemsep}{0pt}\setlength{\parskip}{0pt}}



 


% in-header.tex
% Load required packages (only once)
\usepackage{paracol}
\usepackage[most]{tcolorbox}
\tcbuselibrary{listings, skins, breakable}
\usepackage{tikz}
\usepackage{etoolbox}
\usepackage{fancyhdr}
\usepackage{listings}  % Added listings for syntax highlighting
\usepackage{xcolor}    % for custom colors in listings

% Define tcolorbox style
\tcbset{
  cheatbox/.style={
    enhanced,
    sharp corners,
    colback=white,
    colframe=black,
    coltitle=white,
    colbacktitle=black,
    fonttitle=\bfseries\small,
    boxrule=0.5pt,
    breakable,
    title={},
    leftrule=0.75mm,
    rightrule=0.25mm,
    toprule=0.25mm,
    bottomrule=0.25mm,
    top=4pt,
    bottom=4pt,
    left=6pt,
    right=6pt,
    after skip=6pt,
    before skip=6pt,
  }
}

% Define column-handling commands
\newcommand{\StartCheatColumns}[1]{\begin{paracol}{#1}}
\newcommand{\EndCheatColumns}{\end{paracol}}

% Legacy or alternative environment, if used anywhere
\newenvironment{cheatboxparacol}{
  \par\vspace{2pt}\noindent
  \begin{tcolorbox}[cheatbox]
}{
  \end{tcolorbox}\vspace{2pt}
}

% Define Shaded environment for highlighting
\newtcolorbox[auto counter, number within=section]{Shaded}[2][]{colback=gray!20!white, colframe=black, sharp corners, title=#2,#1}

% Configure listings package
\lstset{ 
  backgroundcolor=\color{gray!20},   % background color
  basicstyle=\ttfamily\footnotesize,   % font style
  keywordstyle=\color{blue}\bfseries,  % keyword color
  commentstyle=\color{green!60!black},% comment color
  stringstyle=\color{red},             % string color
  showstringspaces=false,              % don't show spaces as underscores
  breaklines=true,                     % auto-break lines
  breakatwhitespace=true,
  captionpos=b,                        % bottom caption
  frame=single,                        % single frame around code
  rulecolor=\color{black},             % rule color
}
\makeatletter
\@ifpackageloaded{caption}{}{\usepackage{caption}}
\AtBeginDocument{%
\ifdefined\contentsname
  \renewcommand*\contentsname{Table of contents}
\else
  \newcommand\contentsname{Table of contents}
\fi
\ifdefined\listfigurename
  \renewcommand*\listfigurename{List of Figures}
\else
  \newcommand\listfigurename{List of Figures}
\fi
\ifdefined\listtablename
  \renewcommand*\listtablename{List of Tables}
\else
  \newcommand\listtablename{List of Tables}
\fi
\ifdefined\figurename
  \renewcommand*\figurename{Figure}
\else
  \newcommand\figurename{Figure}
\fi
\ifdefined\tablename
  \renewcommand*\tablename{Table}
\else
  \newcommand\tablename{Table}
\fi
}
\@ifpackageloaded{float}{}{\usepackage{float}}
\floatstyle{ruled}
\@ifundefined{c@chapter}{\newfloat{codelisting}{h}{lop}}{\newfloat{codelisting}{h}{lop}[chapter]}
\floatname{codelisting}{Listing}
\newcommand*\listoflistings{\listof{codelisting}{List of Listings}}
\makeatother
\makeatletter
\makeatother
\makeatletter
\@ifpackageloaded{caption}{}{\usepackage{caption}}
\@ifpackageloaded{subcaption}{}{\usepackage{subcaption}}
\makeatother
\makeatletter
\@ifpackageloaded{tcolorbox}{}{\usepackage[skins,breakable]{tcolorbox}}
\makeatother
\makeatletter
\@ifundefined{shadecolor}{\definecolor{shadecolor}{rgb}{.97, .97, .97}}{}
\makeatother
\makeatletter
\makeatother
\makeatletter
\ifdefined\Shaded\renewenvironment{Shaded}{\begin{tcolorbox}[frame hidden, enhanced, sharp corners, borderline west={3pt}{0pt}{shadecolor}, interior hidden, boxrule=0pt, breakable]}{\end{tcolorbox}}\fi
\makeatother
\usepackage{bookmark}
\IfFileExists{xurl.sty}{\usepackage{xurl}}{} % add URL line breaks if available
\urlstyle{same}
\hypersetup{
  pdftitle={EiBI: Sequenzdistanzen \& globale/semi-globale/lokale Alignments},
  pdfauthor={Gewerd-Strauss},
  colorlinks=true,
  linkcolor={blue},
  filecolor={Maroon},
  citecolor={Blue},
  urlcolor={Blue},
  pdfcreator={LaTeX via pandoc}}


\title{EiBI: Sequenzdistanzen \& globale/semi-globale/lokale Alignments}
\author{Gewerd-Strauss}
\date{}
\begin{document}
% before-body.tex
\newcounter{colnum}
\setcounter{colnum}{1}
\newlength{\colheight}
\setlength{\colheight}{0pt}

% Custom font size for body text

% Custom page geometry

%%%% //FOOTER SETUP %%%%
% Define \Title only if title is present, else empty
\newcommand{\Title}{EiBI: Sequenzdistanzen \&
globale/semi-globale/lokale Alignments}

% Define \Author only if author is present, else empty
\newcommand{\Author}{'Gewerd-Strauss'}

% Define \LastModified only if date is present, else empty
\newcommand{\LastModified}{}

% Helper macro for left footer, uses etoolbox \ifdefempty
\newcommand{\LeftFooter}{%
\ifboolexpr{
  test {\ifdefempty{\Title}} 
  and
  test {\ifdefempty{\LastModified}}
  }
  { % both empty
  % print nothing
  }
  { % at least one nonempty
  \ifdefempty{\Title}
  {\LastModified} % Title empty, print date only
  { % Title nonempty
  \ifdefempty{\LastModified}
  {\Title} % date empty, print title only
  {\Title{} -- \LastModified} % both present, join with --
  }
  }
  }
  
  \AtBeginDocument{%
  \pagestyle{fancy}
  \renewcommand{\headrulewidth}{0pt}

  \fancyhf{}% clear all headers/footers
  
  % Left footer only if LeftFooter nonempty
  \ifdefempty{\LeftFooter}{}{%
  \fancyfoot[L]{\textit{\LeftFooter}}%
  }
  
  % Right footer only if \Author nonempty
  \ifdefempty{\Author}{}{%
  \fancyfoot[R]{\textit{by \Author}}%
  }
  
  \fancyfoot[C]{\thepage}
  }
  
%%%% FOOTER SETUP\\ %%%%

\begin{paracol}{3}

\begin{tcolorbox}[cheatbox, fontupper={\tiny}, fonttitle={\scriptsize}, title={Sequenzdistanzen (1.1)}, ]
\begin{itemize}
\tightlist
\item
  allgemein definiert für ein Alphabet \(\Sigma^{\star}\); besitzt
\item
  Distanz: minimale Anzahl/Kosten von Operationen, um
  \(x\in\Sigma^{\star}\) in \(y\in\Sigma^{\star}\) zu überführen
\item
  Operationen:

  \begin{itemize}
  \tightlist
  \item
    \(C\): Copy von \(x\) zu \(y\)
  \item
    \(S_{c}\forall c\in\Sigma\): substituiere nächsten Buchstaben aus
    \(x\) mit \(c\) in \(y\)
  \item
    \(D\): delete den nächsten Buchstaben aus \(x\) (überspringe ihn)
  \end{itemize}
\end{itemize}
\end{tcolorbox}

\begin{tcolorbox}[cheatbox, fontupper={\tiny}, fonttitle={\scriptsize}, title={...mit Einheitskosten (1.2)}, ]
\begin{align}
d_{\text{Hamming}}(x,y) &= \sum_{i=1}^{n} 1_{x_{i}\neq y_{i}} \\
d_{\text{LCS}}(x,y) &= |x| + |y| - 2|LCS(x,y)| \\
d_{\text{Edit}}(x,y) &= \sum_{min}  a: x\underset{a}{\rightarrow} y | a \in \{C,S_{c},I_{c},D\}
\end{align}

\noindent

\begin{minipage}[t]{0.48\linewidth}
\vspace{0pt}
\begin{tabular}{l|cccc}
 & $C$ & $S_C$ & $I_C$ & $D$ \\
\hline 
Hamming  & 0   & 1     & $\infty$ & $\infty$ \\
LCS      & 0   & $\infty$ & 1 & 1 \\
Edit     & 0 & 1 & 1 & 1
\end{tabular}
\end{minipage}
\hfill
\begin{minipage}[t]{0.48\linewidth}
\vspace{0pt}
\textbf{Notizen:}

- Hamming nur für $|x|=|y|$\\
- Edit: non-trivial, compares $i$-th letter of $x$ with $j$-th letter in $y$\\
- Hamming: trivial, compares $i$-th position of both $x$ and $y$

\end{minipage}
\end{tcolorbox}

\begin{tcolorbox}[cheatbox, fontupper={\tiny}, fonttitle={\scriptsize}, title={Edit-Distanz Details (1.3)}, ]
\begin{itemize}
\tightlist
\item
  kann prozedual oder rekursiv berechnet werden
\item
  \(A_{opt}\) bleibt auch nach Streichen der letzten Alignmentspalte
  optimal: \(A_{opt} = min d(s,t) = min \Sigma_{j} d(s_{j}, t_{j})\)
\item
  Rekursive Berechnung unter Custom-/Einheitskosten (vgl. Tbl in 1.2):
\end{itemize}

\noindent

\begin{minipage}[t]{0.55\linewidth}
\vspace{0pt}
\(
\begin{aligned}
d_{s,t}(i,j)
&= \min \begin{cases}
  \begin{cases}
    d_{s,t}(i-1,j-1)+C & |\, s_i = t_j \\
    d_{s,t}(i-1,j-1)+S_{c} & |\, \text{sonst}
  \end{cases}\\
  d_{s,t}(i-1,j)+D \\
  d_{s,t}(i,j-1)+I_{c} 
\end{cases}
\end{aligned}
\)
\end{minipage}
\vspace{12pt}
\hfill
\begin{minipage}[t]{0.43\linewidth}
\vspace{0pt}
\(
\begin{array}{l}
\text{Init: Row1/Col1:}\\
d_{s,t}(0,0) = 0\\
d_{s,t}(i,0) = i \forall (1 \le i \le |s|)\\
d_{s,t}(0,j) = j \forall (1 \le j \le |t|)
\end{array}
\)
\end{minipage}

\begin{itemize}
\tightlist
\item
  mit dynamischer Programmierung:
\end{itemize}

\noindent

\begin{minipage}[t]{0.55\linewidth}
\vspace{0pt}
\(
\begin{aligned}
M_{i,j}
&= \min \begin{cases}
  \begin{cases}
    M_{i-1,j-1}+C & |\, s_i = t_j \\
    M_{i-1,j-1}+S_{c} & |\, \text{sonst}
  \end{cases}\\
  M_{i-1,j}+D \\
  M_{i,j-1}+I_{c} 
\end{cases}
\end{aligned}
\)
\end{minipage}
\vspace{12pt}
\hfill
\begin{minipage}[t]{0.43\linewidth}
\vspace{0pt}
\(
\begin{array}{l}
\text{Init: Row1/Col1:}\\
M_{0,0}= 0\\
M_{i,0}= i\times I_{c} \forall i\\
M_{0,j}= j\times D \forall j
\end{array}
\)
\end{minipage}
\end{tcolorbox}

\switchcolumn

\begin{tcolorbox}[cheatbox, fontupper={\tiny}, fonttitle={\scriptsize}, title={Globales Alignment via Needleman-Wunsch (3.1)}, colframe=green!100!green,]
\begin{itemize}
\tightlist
\item
  geg: Edit-Matrix und minimale (custom) Edit-Kosten
\item
  ges: Alignment mit minimalen Kosten
\item
  Vorgehen: Stelle Editmatrix gemäß 1.3 (\textbf{dynamisch}) auf; und
  speichere dabei Backtracking-Pointer
\item
  Zurückverfolgung aller co-optimalen Pfade ergibt alle co-optimalen
  Alignments
\item
  Komplexität:

  \begin{itemize}
  \tightlist
  \item
    Zeit: \(O(nm)\) (oder \(O(n^{2})\))
  \item
    Platz: \(O(nm)\)
  \end{itemize}
\end{itemize}
\end{tcolorbox}

\begin{tcolorbox}[cheatbox, fontupper={\tiny}, fonttitle={\scriptsize}, title={Globales Alignment mit affinen Lückenkosten via Gotoh (3.2)}, colframe=green!100!green,]
\begin{itemize}
\tightlist
\item
  affine Lückenkosten: \(g(k) = \alpha + (k-1)\beta\)
\item
  3 Matrizen speichern Kosten für Alignments der Präfixe:

  \begin{itemize}
  \tightlist
  \item
    \(D(i,j)\): (Präfix \((a_{1}\dots a_{i},b_{1}\dots b_{j})\))
  \item
    \(P(i,j)\): (Präfix \((a_{1}\dots a_{i},b_{1}\dots b_{j})\), endet
    in \(b\) mit Lücke \(\begin{pmatrix}a_{i} \\ \_\end{pmatrix}\))
  \item
    \(Q(i,j)\): (Präfix \((a_{1}\dots a_{i},b_{1}\dots b_{j})\), endet
    in \(a\) mit Lücke \(\begin{pmatrix}\_ \\ b_{j}\end{pmatrix}\))
  \end{itemize}
\end{itemize}

\noindent

\begin{minipage}[t]{0.55\linewidth}
\vspace{0pt}
\(
\begin{aligned}
D_{i,j}
&= \min \begin{cases}
  D_{i-1,j-1} + w(a_{i},b_{j})\\
  P_{i,j}\\
  Q_{i,j}
\end{cases} 
\end{aligned}\\
\begin{aligned}
\\\text{with } i,j \geq 1\text{, and for } 1 \leq i \leq |a|, 1 \leq j \leq |b|\text{:}  \\
\end{aligned}\\
\begin{aligned}
P_{i,j}
&= \min \begin{cases}
  D_{i-1,j} + g(1)\\
  P_{i-1,j} + \beta
\end{cases} \\
Q_{i,j}
&= \min \begin{cases}
  D_{i,j-1} + g(1)\\
  Q_{i,j-1} + \beta
\end{cases}
\end{aligned}
\)
\end{minipage}
\vspace{12pt}
\hfill
\begin{minipage}[t]{0.43\linewidth}
\vspace{0pt}
\(
\begin{array}{l}
\text{Init: Row1/Col1:}\\
D(0,0)=0\\
D(i,0)=P(i,0) \\
D(0,j)=Q(0,j) \\
P(0,j)=\infty=Q(i,0)
\end{array}
\)
\end{minipage}

\begin{itemize}
\tightlist
\item
  Zeit- \& Platzkoplexität: \(O(nm)\)
\end{itemize}
\end{tcolorbox}

\switchcolumn

\begin{tcolorbox}[cheatbox, fontupper={\tiny}, fonttitle={\scriptsize}, title={Lokales Alignment via Smith-Waterman Algo mit Scores}, ]
\begin{itemize}
\tightlist
\item
  findet maximum score über gesamte Matrix/gesamten Graphen
\item
  fällt Score unter 0, wird ein neues lokales Alignment begonnen
\end{itemize}
\end{tcolorbox}

\begin{tcolorbox}[cheatbox, fontupper={\tiny}, fonttitle={\scriptsize}, title={Lokales Alignment via globalem Alignment als Graphen mit Ähnlichkeitsmaß}, colframe=yellow!100!yellow,]
Bli
\end{tcolorbox}

\end{paracol}



% after-body.tex
% \EndCheatColumns
\end{document}
